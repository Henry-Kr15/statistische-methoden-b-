% Created 2023-11-13 Mo 15:34
% Intended LaTeX compiler: lualatex
\documentclass[11pt]{article}
\usepackage{graphicx}
\usepackage{longtable}
\usepackage{wrapfig}
\usepackage{rotating}
\usepackage[normalem]{ulem}
\usepackage{amsmath}
\usepackage{amssymb}
\usepackage{capt-of}
\usepackage{hyperref}
\author{Henry Krämerkämper}
\date{\today}
\title{Exercise 4}
\hypersetup{
 pdfauthor={Henry Krämerkämper},
 pdftitle={Exercise 4},
 pdfkeywords={},
 pdfsubject={},
 pdfcreator={Emacs 29.1 (Org mode 9.7)}, 
 pdflang={English}}
\usepackage{biblatex}

\begin{document}

\maketitle
In a lab experiment the following values are measured:

\begin{center}
\includegraphics[width=.9\linewidth]{/home/henry/Arbeitsmappe/Org/.attach/6c/e0ec8e-9fbb-455a-b7f7-1214389861ce/_20231113_113154asymmetrie.png}
\end{center}


The asymmetry values have a measurement value of \(\pm0.011\). The theory says that the
asymmetry is described by an ansatz of the form

\begin{equation*}
   f(\psi) = A_{0} \cos{\psi + \delta}.
\end{equation*}
\section{a}
\label{sec:orgd39a7c6}
Start with the ansatz

\begin{equation*}
   f(\psi) = a_{1} f_{1}(\psi) + a_{2} f_{2} (\psi) = a_{1} \cos(\psi) + a_{2} \sin{(\psi)}
\end{equation*}

and write down the design matrix \textbf{A}.

\begin{center}
\includegraphics[width=.9\linewidth]{/home/henry/Arbeitsmappe/Org/.attach/6c/e0ec8e-9fbb-455a-b7f7-1214389861ce/_20231113_113805Bildschirmfoto vom 2023-11-13 11-37-55.png}
\end{center}

\begin{verbatim}
import numpy as np

psi = np.radians(np.arange(0, 331, 30)) # muss damit numpy nicht weint in rad umgewandelt werden
measurement = np.array([-0.032, 0.010, 0.057, 0.068, 0.076, 0.080, 0.031, 0.005, -0.041, -0.090, -0.088, -0.074])
measurement_error = 0.011

A = np.column_stack([np.cos(psi), np.sin(psi)])
print(f"A = {A}")
\end{verbatim}
\section{b}
\label{sec:orgf34e2f3}

Calculate the solution vector \(\hat{a}\) for the parameters using the method of the least squares.

The solution vector \(\hat{a}\) can be calculated as \(\hat{a} = (\bf{A}^{T} \cdot \bf{A})^{-1} \bf{A}^{T} y\):

\begin{verbatim}
a = np.linalg.inv(A.T @ A) @ A.T @ measurement
print(f"a = {a}")
\end{verbatim}
\section{c}
\label{sec:org6c79f24}

Calculate the covariance matrix \(\bf{V}[\hat{a}]\) as well as the errors of \(a_1\) and \(a_2\) and the correlation coefficient.

The covariance matrix of \(\hat{a}\) is given by \(\bf{V}[\hat{a}] = \sigma^2 (\bf{A}^{T} \cdot \bf{A})^{-1}\), where \(\sigma\) denotes the measurement error.

\begin{verbatim}
V = measurement_error**2 * np.linalg.inv (A.T @ A)
print(f"V = {V}")
\end{verbatim}

The error of \(a_1\) and \(a_2\) can be calculated via the diagonal elements of the covariance matrix:

\begin{verbatim}
a_error = np.sqrt(np.diag(V))
print(f"a_error ={a_error}")
\end{verbatim}
\section{d}
\label{sec:orga166d27}

Calculate \(A_0\) and \(\delta\), their error and the correlation of \(a_1\) and \(a_2\).

\begin{center}
\includegraphics[width=.9\linewidth]{/home/henry/Arbeitsmappe/Org/.attach/6c/e0ec8e-9fbb-455a-b7f7-1214389861ce/_20231113_125723Bildschirmfoto vom 2023-11-13 12-56-57.png}
\end{center}

\begin{verbatim}
A_0 = a[0] * np.sqrt((a[1]**2 / a[0]**2) -1)
delta = np.arctan(-a[1]/a[0])
print(f"A_0 = {A_0}")
print(f"delta = {delta}")
\end{verbatim}

The error of a solution vector \(\vec{y} = f(\vec{x})\) is can be computed by using \(\bf{V}[\vec{y}] = \bf{J} \cdot \bf{V}[\vec{x}] \cdot \bf{J}^{T}\).
Calculate the Jacobian matrix of \(f(\vec{x}) = \begin{bmatrix} A_0 \\ \delta \end{bmatrix}\) :

\begin{center}
\includegraphics[width=.9\linewidth]{/home/henry/Arbeitsmappe/Org/.attach/6c/e0ec8e-9fbb-455a-b7f7-1214389861ce/_20231113_133423Bildschirmfoto vom 2023-11-13 13-34-03.png}
\end{center}

\begin{verbatim}
J = np.array([[(-a[0]/(np.sqrt(a[1]**2 - a[0]**2))), (a[1]/(np.sqrt(a[1]**2 - a[0]**2)))],
              [(a[1]/(a[0]**2 + a[1]**2)),           (-a[0]/(a[0]**2 + a[1]**2))]])

V_2 = J @ V @ J.T
errors = np.sqrt(np.diag(V_2))
# correlation_coefficient = V_2[1,0] / V_2[0,0]
correlation_coefficient = V_2[1,0] / errors[0]**2

print(f"Fehler von A_0: {errors[0]}")
print(f"Fehler von delta: {errors[1]}")
print(f"Korrelationskoeffizient:{correlation_coefficient}")
\end{verbatim}
\end{document}